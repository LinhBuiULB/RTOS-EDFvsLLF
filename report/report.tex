\documentclass{article}

\usepackage[utf8]{inputenc}
\usepackage[pdftex]{graphicx}
\usepackage[left=3cm,right=3cm,top=3cm,bottom=3cm]{geometry}
\usepackage[T1]{fontenc}
\usepackage[francais,english]{babel}
\newcommand*{\escape}[1]{\texttt{\textbackslash#1}}
\frenchbsetup{StandardLists=true}

\usepackage{amsmath}
\usepackage{amssymb}

\usepackage{listings}

%ALGORITHM
\usepackage{algorithm}
\usepackage[noend]{algpseudocode}
\renewcommand{\algorithmicforall}{\textbf{for each}}
\newcommand{\var}[1]{\mathit{#1}}
\newcommand{\func}[1]{\mathrm{#1}}
\algdef{SE}[DOWHILE]{Do}{doWhile}{\algorithmicdo}[1]{\algorithmicwhile\ #1}
%

\usepackage{caption}
\usepackage[hidelinks]{hyperref}
\usepackage{xcolor}
\usepackage{makecell}
\newcommand{\algorithmautorefname}{Algorithm}

\usepackage{graphicx}

\usepackage{fancyhdr}
\pagestyle{fancy}
\fancyhf{}
\fancyhead[R]{\thepage}


\title{[INFO-F404] Real-Time Operating Systems \\ EDF vs LLF - Project report}
\author{\bsc{BUI QUANG PHUONG} Linh - ULB ID : 000427796 \\ \bsc{PAQUET} Michael - ULB ID : 000410753 \\ \bsc{SINGH} Sundeep - ULB ID : 000428022 \\ MA1 Computer Sciences}
\date{December 2018}


\begin{document}

\maketitle

\section*{Introduction}
In this project, we study the difference between two scheduling algorithm on uniprocessors : EDF and LLF. On one side, the practical tools have to been set up such that creating a system generator, parsing information contained in the system file (i.e. periods, WCET and offsets). In the other hand, the main part consisting by implement the EDF and LLF algorithms has to be done based on the system file. 

\section{Pseudo-code and implementation choices}

\subsection{Code structure}
The code is organized in four parts regrouping several functions:  
\begin{itemize}
    \item Parsing functions 
    \item \textit{Question 1} - EDF feasibility interval computation 
    \item \textit{Question 2} - System generator 
    \item \textit{Question 3} - EDF and LLF implementation 
\end{itemize}

\subsubsection{Parsing functions}
The parsing part contains all the functions required to parse the different values (offset, WCET and period) of the system's tasks. This will be useful to manipulate easily those values later.  
\begin{itemize}
    \item {\fontfamily{lmtt}\selectfont readFile(filename)} : reads a file and returns a list of the tasks and its corresponding offset, WCET and periods. 
    \item {\fontfamily{lmtt}\selectfont getOffsetWCETPeriodLists(systemList)} : takes the list of tasks returned by the {\fontfamily{lmtt}\selectfont readFile} function and returns the offset list, the WCET list and the periods list regrouping all the offset/WCET/periods values of all tasks.
\end{itemize}

\subsubsection{EDF feasibility interval computation}
In this part, a function is calculating the EDF feasibility interval while the other one is printing it in the good format as asked in the first question of the statement. Moreover, to compute the feasibility interval, we need to calculate the least common multiple which is done in the {\fontfamily{lmtt}\selectfont LCM} function. 

\begin{itemize}
    \item {\fontfamily{lmtt}\selectfont LCM(numbers)} : calculates the least common multiple of a list of numbers 
    \item {\fontfamily{lmtt}\selectfont computeFeasibilityInterval(newSystemList)} : computes the feasibility interval following the formula :  $[0, max\{D_{i} | i = 1, … , n\}]$ such that $max(D_{i})$ is the greatest deadline value with $i$ the number of tasks.   
    \item {\fontfamily{lmtt}\selectfont printFeasibilityInterval(feasibilityIntervalUpperBound)} : print the feasibility interval following the format : $[0, O_{max} + 2 \cdot P]$
\end{itemize}

\subsubsection{System generator}
This section is dedicated to the generation of systems. To generate a system that satisfies the condition of the utilization percentage, we have first to verify this condition which is done in the {\fontfamily{lmtt}\selectfont matchRequiredUtilisationPercentage} function. Thus, we can easily generate the system file. Note that we arbitrary decided that the limit of the period/WCET value is 50 while that of the offset is 2. Moreover, for the utilisation percentage condition, we have introduced an error margin value which is represented by the parameter {\fontfamily{lmtt}\selectfont delta} in the following functions. In summary, we have : 

\begin{itemize}
    \item {\fontfamily{lmtt}\selectfont matchRequiredUtilisationPercentage(wcets, periods, percentage, delta)} :  checks whether or not we have our required utilization percentage with the values generated randomly
    \item {\fontfamily{lmtt}\selectfont generateTasks(numberOfTasks, requiredUtilisationProcent, delta)} : generates the tasks by returning the offsets, wcet and periods values randomly generated   
    \item {\fontfamily{lmtt}\selectfont printFeasibilityInterval(feasibilityIntervalUpperBound)} : writes the generated tasks in the {\fontfamily{lmtt}\selectfont tasks.txt} file
\end{itemize}

\subsubsection{EDF and LLF implementation}
Before implementing EDF and LLF algorithms, the first thing to do is to get the different task's deadlines. To do that, we simply need to get the multiples of the period's value until the limit of the scheduling taking into account the potential offset. We then store those deadline values in a dictionary doing the link between the deadline and their corresponding tasks illustrated as followed, with $i$ the task number :
$$ \{task_{i}:deadlines_{i}\} $$ 
Furthermore, the most important point of those scheduling algorithms is to know which task is executed first. In the case of EDF, which means \textit{Earliest Deadline First}, we pick the lowest deadline at time $t$ as indicated by his name. Thus, we need a function which is doing this job, that's why {\fontfamily{lmtt}\selectfont getSmallestDeadlines} is present. In the case of LLF, which means \textit{Least Laxity First}, a new notion appears : the laxity. It's this new notion of laxity that will determinate the task to execute at time $t$. The laxity of a job $j$ is computable by the formula $$ l_{j}(t) = d - t - ( e - \epsilon_{j}(t)) $$  where $d$ is the task's deadline, $t$ the current time, $e$ the execution time (WCET) and $\epsilon_{j}(t)$ the cumulative CPU time used by $J$.

To summarize, those functions are added to those previously presented to finally implement the EDF and LLF algorithms, first, to get the deadlines : 
\begin{itemize}
    \item {\fontfamily{lmtt}\selectfont getMultiplesOf(number, limit, offset)} :  computes the multiples of a number considering the offset until a certain limit. Here this number is the first deadline of a task and the multiples are the next deadlines. 
    \item {\fontfamily{lmtt}\selectfont getTasksDeadlines(systemList, upperBound, offsets)} :  returns a dictionary linking the tasks of the system and all their respective deadlines
\end{itemize}

Now that we have all the deadlines, here are the two functions which compute the smallest deadline in case of EDF and the laxities in the case of LLF:
\begin{itemize}
    \item {\fontfamily{lmtt}\selectfont getSmallestDeadlines(tasksDeadlinesDict, isJobDoneUntilNextDeadline)} :  get the task with the smallest deadline value in the previous computed dictionary in {\fontfamily{lmtt}\selectfont getTasksDeadlines(\\systemList, upperBound, offsets)}. 
    \item {\fontfamily{lmtt}\selectfont computeLaxities(time, tasksDeadlines, e, isJobDoneUntilNextDeadline, CPUTimeUsed)} : computes the laxity of all jobs to later get the smallest value to execute in the laxity list returned by this function. 
\end{itemize}

Moreover, some utility functions are created to make the code more clear such that initialization functions or conditions checking functions:
\begin{itemize}
    \item {\fontfamily{lmtt}\selectfont initJobsList(systemList)} :  initialiaze a dictionary with $i$ zero, where $i$ is the number of tasks. Those values are the current job number of the tasks, if the current job executed of the first task, then the value of the first key of the dictionary will be 1. 
    \item {\fontfamily{lmtt}\selectfont initIsJobDoneDict(systemList)} : initialize a dictionary of $i$ boolean values indicating if the current job is done, where $i$ is the number of tasks
    \item {\fontfamily{lmtt}\selectfont isSchedulable(systemList, end)} : check whether or not a system is schedulable depending of the feasibility interval 
    \item {\fontfamily{lmtt}\selectfont isDeadlineMissed(deadline, t)} : check whether or not a deadline is missed, happens when a job is not fully executed before his deadline 
\end{itemize}

All the tools are now available to implement the EDF and LLF algorithms, \textbf{the pseudo-code of the two following functions are detailed in the next section}. 
\begin{itemize}
    \item {\fontfamily{lmtt}\selectfont EDF(system, begin, end)} : the "earliest deadline first" scheduling algorithm
    \item {\fontfamily{lmtt}\selectfont LLF(system, begin, end)} : the "least laxity first" scheduling algorithm
\end{itemize}

Finally, a clear command line or graphical output are available. This printing part is done in those two 
following functions : 

\begin{itemize}
    \item {\fontfamily{lmtt}\selectfont printOutputs(tasksExecuted, arrivalJobOutput, begin, end, systemList, preemptionsNb)} :  prints the scheduling in the command line 
    \item {\fontfamily{lmtt}\selectfont printGraph(tasksExecuted)} : displays the graphical output of the scheduling   
\end{itemize}

\newpage
\subsection{Pseudo-code EDF and LLF}

Our implementation of the EDF scheduling algorithm is presented in the pseudo-code presented in \autoref{alg:edf} while the pseudo-code for LLF is presented in \autoref{alg:llf}. The difference of LFF compared to EDF are brought out in \textcolor{red}{red}. 

\vspace{5cm}
\begin{algorithm} 
  \caption{EDF scheduling} \label{alg:edf}
  \begin{algorithmic}[1]
  \Procedure{EDF}{system, begin, end}
  \State Parsing of the system file
  \State Initialization of $tasksDeadlinesDict$ : the tasks deadlines dictionary, and $jobs$ : tasks jobs list 
  \State Initialization of empty list $tasksExecuted$ which will contain the tasks executed over time $t$ \\
  \If {schedulable} 
  \Comment {end $\le$ upper bound of feasibility interval}
  \While {the scheduling isn't finished}
  \ForAll {deadline $i$}
  \If {$i$ = currentTime $t$}
  \State job arrival 
  \EndIf 
  \EndFor
  \\
  \State Assign the current executing task $j$ by taking the task whose the current job 
  \State has the \textbf{smallest deadline} 
  \\
  \If{deadline missed}
  \Comment {deadline of $job_{j}$ < $t$}
  \State break the loop and add "miss" to $tasksExecuted$
  \Else
  \State Append $j$ to the list of task executed 
  \State Decrements $WCET$ of $j$ by 1 
  \State Compute preemption number
  \If {current job of $j$ done}
  \Comment {$WCET_{j}$ = 0}
  \State Increment $jobs_{j}$ by 1 
  \State Remove the deadline of $job_{j}$ from $tasksDeadlinesDict$ 
  \EndIf
  \EndIf
  \EndWhile
  \EndIf
  \EndProcedure
  \end{algorithmic}
\end{algorithm}

\begin{algorithm}[H]
  \caption{LLF scheduling} \label{alg:llf}
  \begin{algorithmic}[1]
  \Procedure{LLF}{system, begin, end}
  \State Parsing of the system file
  \State Initialization of $tasksDeadlinesDict$ : the tasks deadlines dictionary, and $jobs$ : tasks jobs list 
  \State Initialization of empty list $tasksExecuted$ which will contain the tasks executed over time $t$
  \State \textcolor{red}{Initialization of empty list $laxityOfJobs$ which will contain the computed laxities for all jobs}
  \\
  \If {schedulable} 
  \Comment {end $\le$ upper bound of feasibility interval}
  \While {the scheduling isn't finished}
  \ForAll {deadline $i$}
  \If {$i$ = currentTime $t$}
  \State job arrival 
  \EndIf 
  \EndFor
  \State \textcolor{red}{Compute the laxity of jobs and put it in $laxityOfJobs$}
  \\
  \State Assign the current executing task $j$ by taking the task whose the current job 
  \State has the \textcolor{red}{\textbf{smallest laxity}}  
  \\
  \If{deadline missed}
  \Comment {\textcolor{red}{min(laxityOfJobs) < 0}}
  \State break the loop and add "miss" to $tasksExecuted$
  \Else
  \State Append the current executing task $j$ to list of task executed 
  \State Decrements $WCET$ of $j$ by 1 
  \State Compute preemption number
  \If {current job of $j$ done}
  \Comment {$WCET_{j}$ = 0}
  \State Increment $jobs_{j}$ by 1 
  \State Remove the deadline of $job_{j}$ from $tasksDeadlinesDict$ 
  \EndIf
  \EndIf
  \EndWhile
  \EndIf
  \EndProcedure
  \end{algorithmic}
\end{algorithm}


\section{Difficulties encountered}
No real difficulties were encountered. The main difficulties are generally related to the Graphical User Interface which was then problems with the corresponding library {\fontfamily{lmtt}\selectfont matplotlib}. To resolve this, we simply do some researches associated the problems encountered.

\section{Graphical part library}
To plot the outcome of the scheduling graphically, we used the library {\fontfamily{lmtt}\selectfont matplotlib} of {\fontfamily{lmtt}\selectfont Python}. No download is required. The only requirement is to installing the library in your computer using in the command line : 
\begin{itemize}
    \item in Windows : {\fontfamily{lmtt}\selectfont pip install matplotlib}
    \item in Linux/MacOS : {\fontfamily{lmtt}\selectfont sudo apt-get build-dep python-matplotlib} 
\end{itemize}

\section{EDF vs LLF : preemption comparison}

The difference observed between the number of preemption for an EDF scheduling and an LLF scheduling is that LLF has quite more preemptions than EDF. This difference is explained by the dynamic of priority change for each case. Indeed, the priorities of EDF are dynamic at task level and fixed at job level while LLF is a job-level dynamic priority scheduler. Therefore, there are more priorities change in case of LLF than EDF, then more risks of preemption when scheduling LLF.

\end{document}